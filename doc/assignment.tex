% !TEX TS-program = pdflatex
% !TEX encoding = UTF-8 Unicode

% This is a simple template for a LaTeX document using the "article" class.
% See "book", "report", "letter" for other types of document.

\documentclass[11pt]{article} % use larger type; default would be 10pt

\usepackage[utf8]{inputenc} % set input encoding (not needed with XeLaTeX)

%%% Examples of Article customizations
% These packages are optional, depending whether you want the features they provide.
% See the LaTeX Companion or other references for full information.

%%% PAGE DIMENSIONS
\usepackage{geometry} % to change the page dimensions
\geometry{a4paper} % or letterpaper (US) or a5paper or....
% \geometry{margin=2in} % for example, change the margins to 2 inches all round
% \geometry{landscape} % set up the page for landscape
%   read geometry.pdf for detailed page layout information

\usepackage{graphicx} % support the \includegraphics command and options

% \usepackage[parfill]{parskip} % Activate to begin paragraphs with an empty line rather than an indent

%%% PACKAGES
\usepackage{booktabs} % for much better looking tables
\usepackage{array} % for better arrays (eg matrices) in maths
\usepackage{paralist} % very flexible & customisable lists (eg. enumerate/itemize, etc.)
\usepackage{verbatim} % adds environment for commenting out blocks of text & for better verbatim
\usepackage{subfig} % make it possible to include more than one captioned figure/table in a single float
% These packages are all incorporated in the memoir class to one degree or another...
\usepackage{pdfpages}
\usepackage{amsmath,mathtools,amsfonts}
\usepackage{bm}
\usepackage{listings}\usepackage{color}

\definecolor{mygreen}{rgb}{0,0.6,0}
\definecolor{mygray}{rgb}{0.5,0.5,0.5}
\definecolor{mymauve}{rgb}{0.58,0,0.82}

\lstset{ 
	backgroundcolor=\color{white},   % choose the background color; you must add \usepackage{color} or \usepackage{xcolor}; should come as last argument
	basicstyle=\footnotesize,        % the size of the fonts that are used for the code
	breakatwhitespace=false,         % sets if automatic breaks should only happen at whitespace
	breaklines=true,                 % sets automatic line breaking
	captionpos=b,                    % sets the caption-position to bottom
	commentstyle=\color{mygreen},    % comment style
	deletekeywords={...},            % if you want to delete keywords from the given language
	escapeinside={\%*}{*)},          % if you want to add LaTeX within your code
	extendedchars=true,              % lets you use non-ASCII characters; for 8-bits encodings only, does not work with UTF-8
	firstnumber=001,                % start line enumeration with line 1000
	frame=single,	                   % adds a frame around the code
	keepspaces=true,                 % keeps spaces in text, useful for keeping indentation of code (possibly needs columns=flexible)
	keywordstyle=\color{blue},       % keyword style
	language=Octave,                 % the language of the code
	morekeywords={*,...},            % if you want to add more keywords to the set
	numbers=left,                    % where to put the line-numbers; possible values are (none, left, right)
	numbersep=5pt,                   % how far the line-numbers are from the code
	numberstyle=\tiny\color{mygray}, % the style that is used for the line-numbers
	rulecolor=\color{black},         % if not set, the frame-color may be changed on line-breaks within not-black text (e.g. comments (green here))
	showspaces=false,                % show spaces everywhere adding particular underscores; it overrides 'showstringspaces'
	showstringspaces=false,          % underline spaces within strings only
	showtabs=false,                  % show tabs within strings adding particular underscores
	stepnumber=2,                    % the step between two line-numbers. If it's 1, each line will be numbered
	stringstyle=\color{mymauve},     % string literal style
	tabsize=2,	                   % sets default tabsize to 2 spaces
	title=\lstname                   % show the filename of files included with \lstinputlisting; also try caption instead of title
}

%%% HEADERS & FOOTERS
\usepackage{fancyhdr} % This should be set AFTER setting up the page geometry
\pagestyle{fancy} % options: empty , plain , fancy
\renewcommand{\headrulewidth}{1pt} % customise the layout...
\lhead{Jesse Sheehan (53366509)}\chead{}\rhead{Will Cowper (81163265)}
\lfoot{}\cfoot{\thepage}\rfoot{}

%%% SECTION TITLE APPEARANCE
%\usepackage{sectsty}
%\allsectionsfont{\sffamily\mdseries\upshape} % (See the fntguide.pdf for font help)
% (This matches ConTeXt defaults)

%%% END Article customizations
\makeatletter
\renewcommand*\env@matrix[1][*\c@MaxMatrixCols c]{%
  \hskip -\arraycolsep
  \let\@ifnextchar\new@ifnextchar
  \array{#1}}
\makeatother

\newenvironment{sysmatrix}[1]
 {\left(\begin{array}{@{}#1@{}}}
 {\end{array}\right)}
\newcommand{\ro}[1]{%
  \xrightarrow{\mathmakebox[\rowidth]{#1}}%
}
\newlength{\rowidth}% row operation width
\AtBeginDocument{\setlength{\rowidth}{3em}}

\title{COSC364 RIPv2 Assignment}
\date{\today}
\author{Jesse Sheehan (53366509)\\ Will Cowper (81163265)}

\begin{document}

\maketitle

\tableofcontents

\newpage

\includepdf[pages=1]{includes/plagiarism.pdf}

\section{Questions}

As required, the following questions have been answered:

\subsection{Contribution}
The contribution toward the entire project was an even split. Both partners felt as though the work they had contributed was worth 50\% each.

\subsection{Reflection}

% Which aspects of your overall program do you consider to be particularly well done?
Some of the smaller modules in our codebase have been implemented quite well. For example, the Timer and Bencode modules have a very focused purpose and were discrete enough to be able to be doctested. We found that making use of recursion in the Bencode module prevented the complexity of its functions from becoming too high. The Timer modlue has many more features that what are being used in our project. It turned out to be too fully-featured and we could have done just as well with a stripped-back version of the same thing. We also had a good user-interface that clearly displays the current routing table. Our protocol module contains functions and classes that deal with preparing data for transmission, this includes a checksum that ensures that the data arrives intact or not at all.


% Which aspects of your overall program could be improved?
The overall system design could be improved. We rewrote some modules several times in order to get it to feel as though it would be easy to work with. If we were to improve upon the curent design, we could add more features, such as accepting keyboard commands while the router is running, etc.

\subsection{Event Processing}
% How have you ensured the atomicity of event processing?
Our entire program is based around a main loop that waits for incoming packets and if it doesn't receive any, it will do other things, such as updating the timers, updating the routing table, rendering the screen, etc. We use lists to ensure that our incoming packets are serviced in the order in which they arrive. When packets are processed, they may trigger updates to the routers neighbors. These updates are serviced after the periodic updates have finished being received. Once these updates have been sent to its neighbors, the router simply waits for more information to arrive.

By waiting for events, and servicing other processes when there are no events, we can make use of the router's time more efficiently.


\subsection{Testing}
Many of the smaller functions in the project were discrete enough that we could use doctests on them although we found that as the project grew, the complex dependencies between the objects also grew. Once many functions in a module were working well and passing doctests, we were also able to test the module as a whole, by using many functions together and testing them. This led to us discovering that some of our functions weren't returning the correct values or weren't accepting the right parameters, etc. After getting an entire module working properly, we focussed on combining several modules together in a new module and testing that. Because of the amount of things we needed to test we didn't bother with writing any test cases. This was perhaps a poor decision in terms of testing.

Eventually we had most of the program working and had to create test configuration files for the router program to use. We found this process to be very tedious and so, wrote a program to generate these files for us. Our testing became much easier after this as we didn't have to manually write these configuration files.

\newpage
\section{Appendices}

\subsection{Source Code}

\subsubsection{src/\_\_main\_\_.py}
\lstinputlisting[language=Python]{../src/__main__.py}

\subsubsection{src/bencode.py}
\lstinputlisting[language=Python]{../src/bencode.py}

\subsubsection{src/config.py}
\lstinputlisting[language=Python]{../src/config.py}

\subsubsection{src/protocol.py}
\lstinputlisting[language=Python]{../src/protocol.py}

\subsubsection{src/routing\_table\_entry.py}
\lstinputlisting[language=Python]{../src/routing_table_entry.py}

\subsubsection{src/routing\_table.py}
\lstinputlisting[language=Python]{../src/routing_table.py}

\subsubsection{src/server.py}
\lstinputlisting[language=Python]{../src/server.py}

\subsubsection{src/timer.py}
\lstinputlisting[language=Python]{../src/timer.py}

\subsubsection{src/utils.py}
\lstinputlisting[language=Python]{../src/utils.py}

\newpage
\subsection{Configuration Files}

\subsubsection{configs/networks/figure1/1.conf}
\lstinputlisting{../configs/networks/figure1/1.conf}

\subsubsection{configs/networks/figure1/2.conf}
\lstinputlisting{../configs/networks/figure1/2.conf}

\subsubsection{configs/networks/figure1/3.conf}
\lstinputlisting{../configs/networks/figure1/3.conf}

\subsubsection{configs/networks/figure1/4.conf}
\lstinputlisting{../configs/networks/figure1/4.conf}

\subsubsection{configs/networks/figure1/5.conf}
\lstinputlisting{../configs/networks/figure1/5.conf}

\subsubsection{configs/networks/figure1/6.conf}
\lstinputlisting{../configs/networks/figure1/6.conf}

\subsubsection{configs/networks/figure1/7.conf}
\lstinputlisting{../configs/networks/figure1/7.conf}

\newpage
\subsection{Other Files}

\subsubsection{tools/generate\_network.py}
The following file will interactively prompt the user for information about a network. It will then create all the necessary configuration files for the network to run.
\lstinputlisting[language=Python]{../tools/generate_network.py}

\end{document}
